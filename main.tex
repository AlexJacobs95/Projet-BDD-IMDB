\documentclass[10pt,a4paper]{article}
\usepackage[left=3cm,right=3cm,top=3cm,bottom=3cm]{geometry}
\usepackage[utf8]{inputenc}
\usepackage[T1]{fontenc}
\usepackage[export]{adjustbox}
\usepackage[french]{babel}
\usepackage{graphicx}
\usepackage{eso-pic}
\usepackage{transparent}
\usepackage[export]{adjustbox}
\usepackage{hyperref} \hypersetup{colorlinks, citecolor=black, filecolor=black, linkcolor=black, urlcolor=black,}
\usepackage[nonumberlist]{glossaries}
\usepackage{glossaries}
\usepackage{imakeidx}

% 1 : Logo ulb en fond
\newcommand\BackgroundPic{%
    \put(0,-47){%
        \parbox[b][\paperheight]{\paperwidth}{%
            \vfill
            \centering
            {\transparent{0.09}\includegraphics[width=1.17 \textwidth]{logo_ulb.jpg}}%
            \vfill
        }
    }
}
\begin{document}
\AddToShipoutPicture*{\BackgroundPic}

\begin{titlepage}
\begin{center}
\vspace*{-1.5cm}
\includegraphics[width=8cm]{ULB.jpg}
\vspace{6cm}

\par{\huge \textbf{Rapport Projet}}\bigbreak
\bigbreak
\par{\huge \textbf{\textit{{IMDB: Internet Movie Database}}}}
\vspace{2cm}
\par{\large [Cours: INFO-H-303]}
\vspace{2cm}

\par \hrulefill \par
\vspace{1cm}
\bsc{Jacobs} Alexandre 
\bsc{\&}
\bsc{Engelman} David
\bsc{\&}
\bsc{Engelman} Benjamin
{\emph \\BA2 Informatique}
\vspace{0.7cm}
\par \hrulefill \par

\vspace{2cm}
\par Mai 2017

\end{center}
\end{titlepage}
\newpage
\tableofcontents
\newpage

\section{Introduction}
Dans le cadre du cours de "Base de donnée", nous avons du réaliser un projet mettant en applications les différents concepts vus en cours et lors des séances de travaux pratiques.
\subsection{But du projet}
Ici le but fût à partir de fichiers disponible sur internet de récréer un base de donnée sur le concept de "IMDB: Internet Movie Database". Le projet fut divisé en deux grandes parties, la première partie (section \ref{première partie}) étant la création d'un diagramme entité-association et un diagramme relationnel. La seconde partie (section \ref{deuxieme partie}) fut la plus pratique qui était l'implémentation de la base de donnée. \\

En plus de devoir faire une base de donnée, nous avons dû faire les scripts qui permettaient la création de la base de donnée et l'extraction des données qui y sont stockés. Aussi à fin de pouvoir utiliser notre base de donnée, nous avons crée un site web permettant de rechercher sur des critère simple ou avancés des oeuvres(film, série ou épisodes), des personnes(acteur, directeur, auteur).

\section{Première Partie}\label{première partie}

\subsection{Modèle entité-association}

\subsection{Modèle relationnel}

\section{Deuxième Partie}\label{deuxieme partie}

\subsection{Méthode d'extraction des données}

\subsection{Requêtes demandées}
\subsubsection{SQL}
\subsubsection{Algèbre relationnel}
\subsubsection{Calcul relationnel tuple}

\section{Conclusion}
\end{document}