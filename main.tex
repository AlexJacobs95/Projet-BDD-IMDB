\documentclass[10pt,a4paper]{article}
\usepackage[left=1.5cm,right=1.5cm,top=3cm,bottom=3cm]{geometry}
\usepackage[utf8]{inputenc}
\usepackage[T1]{fontenc}
\usepackage[export]{adjustbox}
\usepackage[french]{babel}
\usepackage{graphicx}
\usepackage{eso-pic}
\usepackage{transparent}
\usepackage[export]{adjustbox}
\usepackage{hyperref} \hypersetup{colorlinks, citecolor=black, filecolor=black, linkcolor=black, urlcolor=black,}
\usepackage{xcolor,listings}
\usepackage{textcomp}
\lstset{upquote=true}

% 1 : Logo ulb en fond
\newcommand\BackgroundPic{%
    \put(0,-47){%
        \parbox[b][\paperheight]{\paperwidth}{%
            \vfill
            \centering
            {\transparent{0.09}\includegraphics[width=1.17 \textwidth]{logo_ulb.jpg}}%
            \vfill
        }
    }
}
\begin{document}
\AddToShipoutPicture*{\BackgroundPic}

\begin{titlepage}
\begin{center}
\vspace*{-1.5cm}
\includegraphics[width=8cm]{ULB.jpg}
\vspace{6cm}

\par{\huge \textbf{Rapport Projet}}\bigbreak
\bigbreak
\par{\huge \textbf{\textit{{IMDB: Internet Movie Database}}}}
\vspace{2cm}
\par{\large [Cours: INFO-H-303]}
\vspace{2cm}

\par \hrulefill \par
\vspace{1cm}
\bsc{Jacobs} Alexandre 
\bsc{\&}
\bsc{Engelman} David
\bsc{\&}
\bsc{Engelman} Benjamin
{\emph \\BA2 Informatique}
\vspace{0.7cm}
\par \hrulefill \par

\vspace{2cm}
\par Mai 2017

\end{center}
\end{titlepage}
\newpage
\tableofcontents
\newpage

\section{Introduction}
Dans le cadre du cours de "Base de donnée", nous avons du réaliser un projet mettant en applications les différents concepts vus en cours et lors des séances de travaux pratiques.
\subsection{But du projet}
Ici le but fût à partir de fichiers disponible sur internet de récréer un base de donnée sur le concept de "IMDB: Internet Movie Database". Le projet fut divisé en deux grandes parties, la première partie (section \ref{première partie}) étant la création d'un diagramme entité-association et un diagramme relationnel. La seconde partie (section \ref{deuxieme partie}) fut la plus pratique qui était l'implémentation de la base de donnée et la création du site web. \\

En plus de devoir faire une base de donnée, nous avons dû faire les scripts qui permettaient la création de la base de donnée et l'extraction des données qui y sont stockés. Aussi à fin de pouvoir utiliser notre base de donnée, nous avons crée un site web permettant de rechercher sur des critère simple ou avancés des œuvres(film, série ou épisodes), des personnes(acteur, directeur, auteur).

\newpage
\section{Première Partie}\label{première partie}

\subsection{Modèle entité-association}

\includegraphics[width = 17cm]{Diag_E-A}

\subsubsection{Contraintes et hypothèses}

\noindent\underline{Contraintes}
\\ \\ L'année de fin d'une série doît être postérieur à l'année de commencement de la série. Un numéro d’épisode ne peut pas être négaif.
La date de sortie d’un épisode doit être postérieure ou égale à la date de sortie de la série auquel il appartient.
\\
\\
\noindent\underline{Hypothèses}
\\ \\ Une personne (Auteur, Producteur, Directeur, Acteur) peut n’avoir participé à aucune œuvre.
Une oeuve peut avoir été tournée dans plusieurs pays, avoir plusieurs langues et avoir différents genres.
Une saison possède au moins un épisode.
L’année de sortie d’une œuvre peut être inconnue.
Si un pays se trouve dans la base de donnée c’est qu’au moins une Œuvre y a été tournée.
Si un genre se trouve dans la base de donnée c’est qu’il existe au moins une Œuvre de ce genre.
Si une langue se trouve dans la base de donnée, c’est qu’elle est parlée dans au moins une œuvre.
Un acteur peut jouer plusieurs rôles dans une même œuvre.
Un rôle peut être jouer plusieurs acteurs (exemple: enfant et adulte).
Plusieurs personnes peuvent avoir le même nom et le même prénom, dans ce cas nous leur attribuons également un numéro pour les différencier.
Le nom ou le prénom d’une personne peut être inconnu mais pas les deux en même temps.

\subsection{Modèle relationnel}

\textbf{Personne}(\underline{Nom, Prénom, Numéro}, Genre)
\\
\\
\textbf{Auteur}(\underline{Nom, Prénom, Numéro})
\\
		Auteur.(Nom, Prénom, Numéro) référence Personne.(Nom, Prénom, Numéro)
\\ \\
\textbf{AuteurOeuvre}(\underline{Nom, Prénom, Numéro, Référence})
\\
		AuteurOeuvre.(Nom, Prénom, Numéro) référence Auteur.(Nom, Prénom, Numéro)
\\
		AuteurOeuvre.Référence référence Œuvre.Référence
\\
\\
\textbf{Directeur}(\underline{Nom, Prénom, Numéro}, Genre)
\\
		Directeur.(Nom, Prénom, Numéro, Genre) référence Personne.(Nom, Prénom, Numéro, Genre)
\\ \\
\textbf{DirecteurOeuvre}(\underline{Nom, Prénom, Numéro, Référence})
\\
		DirecteurOeuvre.(Nom, Prénom, Numéro) référence Directeur.(Nom, Prénom, Numéro)
        \\
		DirecteurOeuvre.Référence référence Œuvre.Référence
\\
\\
\textbf{Acteur}(\underline{Nom, Prénom, Numéro}, Genre)
\\
		Acteur.(Nom, Prénom, Numéro) référence Personne.(Nom, Prénom, Numéro)
\\
\\
\textbf{ActeurRole}(\underline{Nom, Prénom, Numéro, Rôle})
\\
		ActeurRole.(Nom, Prénom, Numéro) référence Acteur.(Nom, Prénom, Numéro)
    \\
		ActeurRole.(Role) référence Rôle.Rôle
\\
\\
\textbf{Rôle}(\underline{Rôle, Nom, Prénom, Référence}, RéférenceO)
\\
		Rôle.RéférenceO référence Œuvre.Référene
\\
\\
\textbf{Œuvre}(\underline{Référence}, Titre, Note, AnnéeSortie)
\\
\\
\textbf{Pays}(\underline{Nom})
\\
\\
\textbf{PaysOeuvre}(\underline{Nom, Référence})
\\
		Nom référence Pays.Nom
        \\
		Référence référence Œuvre.Référence
\\ 
\\
\textbf{Genre}(\underline{Nom})
\\
\\
\textbf{GenreOeuvre}(\underline{Nom, Référence})
\\
		Nom référence Genre.Nom
    \\
		Référence référence Œuvre.Référence
\\
\\
\textbf{Langue}(\underline{Nom})
\\
\\
\textbf{LangueOeuvre}(\underline{Nom, Référence})
\\
	Nom référence Langue.Nom
    \\
	Référence référence Œuvre.Référence
\\
\\
\textbf{Film}(\underline{Référence})
\\
	Film.Référence référence Œuvre. Référence
\\
\\
\textbf{Série}(\underline{Référence}, AnnéeFin)
\\
	Série.Référence référence Œuvre.Référence
\\
\\
\textbf{Episode}(\underline{Référence},TitreE, NuméroE, Date de sortie, RéférenceS)
\\
		Episode.Référence référence Œuvre.référence
        \\
		Epidode.RéférenceS référence Saison.Référence
\\
\\
\textbf{Administrateur}(\underline{Adresse mail}, mot de passe)
\subsubsection{Contraintes modèle relationnel}
\noindent Episode.Date de sortie >= Serie.Date de sortie \\
RéférenceS est unique dans la relation saison (une saison ne peut appartenir qu’à une série) \\
RéférenceS est unique dans la relation Episode (un Episode ne peut appartenir qu’à une saison) \\
Référence est unique dans la relation Rôle (un Rôle ne peut appartenir qu’à une Œuvre)

\newpage
\section{Deuxième Partie}\label{deuxieme partie}

\subsection{Méthode d'extraction des données}

Pour l'extraction des données qui seront insérer dans notre base de donnée. Nous avons tout d'abord réaliser un script bash permettant le téléchargement des fichiers qui nous étaient nécessaires. En ce qui concerne le traitement des fichiers télécharger, nous avons réaliser un autre script de bash qui lui fait appel à du code python permettant de formaté les fichiers selon un standard que nous nous sommes fixé pour permettre l'insertion des données dans la base de donnée plus simple. Nous avons donc réaliser différents parsers avec un "BaseParser" contenant des méthodes de base pour tout les autres parsers.

\subsection{Requêtes demandées}
Les requêtes demandées étaient les suivantes:
\begin{list}{-}{}
  \item R1 : Les acteurs qui ont joué toutes les années entre 2003 et
    2007.
  \item R2 : Les auteurs qui ont écrit au moins deux films pendant la
    même année.
  \item R3 : Les acteurs Y qui sont à une distance 2 d'un acteur X. Un
    acteur Y est à une distance 1 d'un acteur X si ces deux acteurs
    ont joué dans le même film.
  \item R4 : Les épisodes de série où il n'y a aucun acteur masculin.
  \item R5 : Les acteurs qui ont joué dans le plus de séries.
  \item R6 : Les séries avec leur nombre total d'épisodes, le nombre
    d'épisodes moyen par an et le nombre d'acteurs moyen par saison
    depuis leur année de création et ce pour toutes les séries dont la
    note est supérieure à la moyenne des notes des séries.
\end{list}
\subsubsection{SQL}
\definecolor{backcolour}{rgb}{0.95,0.95,0.92}
\begin{lstlisting}[
           language=SQL,
           backgroundcolor=\color{backcolour},
           showspaces=false,
           basicstyle=\ttfamily,
           keepspaces=true,                 
           numbers=left,                    
           numbersep=5pt,
           numberstyle=\tiny,
           commentstyle=\color{blue},
           caption = Requête 1,
           captionpos = b, 
           tabsize = 4,
           keywordstyle=\color{magenta},
           breakatwhitespace=false,         
           breaklines=true,
        ]
/*R1*/
SELECT Nom, Prenom, Numero
FROM Acteur a
WHERE (SELECT count(distinct AnneeSortie)
	   FROM (
			SELECT AnneeSortie, Nom, Prenom, Numero
			FROM Oeuvre o INNER JOIN Role r
			ON ID = OID
			WHERE AnneeSortie BETWEEN 2003 AND 2007) t
	   WHERE t.Prenom = a.Prenom and t.Nom = a.Nom and t.Numero = a.Numero) = 5;
       
/*R1 bis*/:
SELECT Nom, Prenom, Numero
FROM (
	SELECT AnneeSortie, Nom, Prenom, Numero
	FROM Oeuvre o INNER JOIN Role r
	ON ID = OID
	WHERE AnneeSortie BETWEEN 2003 AND 2007) t
group by Nom, Prenom, Numero
having count(distinct AnneeSortie) = 5;
\end{lstlisting}
\newpage
\begin{lstlisting}[
           language=SQL,
           backgroundcolor=\color{backcolour},
           showspaces=false,
           basicstyle=\ttfamily,
           keepspaces=true,                 
           numbers=left,                    
           numbersep=5pt,
           numberstyle=\tiny,
           commentstyle=\color{blue},
           caption = Requête 2,
           captionpos = b,
           tabsize = 4,
           keywordstyle=\color{magenta},
           breakatwhitespace=false,         
           breaklines=true,
        ]
/*R2*/
SELECT distinct Nom, Prenom, Numero
FROM(
	SELECT AnneeSortie, Nom,Prenom, Numero
	FROM Oeuvre o
	INNER JOIN Film f ON o.ID = f.FilmID
	INNER JOIN EcritPar ON ID = OID
	group by AnneeSortie, Nom, Prenom, Numero
	having count(*) >=2 )t;
\end{lstlisting}

\begin{lstlisting}[
           language=SQL,
           backgroundcolor=\color{backcolour},
           showspaces=false,
           basicstyle=\ttfamily,
           keepspaces=true,                 
           numbers=left,                    
           numbersep=5pt,
           numberstyle=\tiny,
           commentstyle=\color{blue},
           caption = Requête 3,
           captionpos = b,
           tabsize = 4,
           keywordstyle=\color{magenta},
           breakatwhitespace=false,         
           breaklines=true,
        ]
/*R3*/
SELECT distinct Nom, Prenom, Numero
FROM(
	SELECT OID
	FROM(
		SELECT 	R2.OID
		FROM(
			SELECT T2.Nom, T2.Prenom, T2.Numero, T2.OID
			FROM(
					SELECT R1.Prenom, R1.Nom, R1.Numero, R1.OID
					FROM (
						SELECT *
						FROM Role
						WHERE Nom = 'Willis' and Prenom = 'Bruce' and Numero = 'NA') as T
						JOIN Role R1 ON T.OID = R1.OID) as T2
			INNER JOIN Film F on F.FilmID = T2.OID) as T3
		JOIN Role R2 ON T3.Prenom = R2.Prenom AND T3.Nom = R2.Nom AND T3.Numero = R2.Numero) as T4
	INNER JOIN Film F2 on F2.FilmID = T4.OID) as T5
JOIN Role R3 ON T5.OID = R3.OID;
\end{lstlisting}

\begin{lstlisting}[
           language=SQL,
           backgroundcolor=\color{backcolour},
           showspaces=false,
           basicstyle=\ttfamily,
           keepspaces=true,                 
           numbers=left,                    
           numbersep=5pt,
           numberstyle=\tiny,
           commentstyle=\color{blue},
           caption = Requête 4,
           captionpos = b,
           tabsize = 4,
           breakatwhitespace=false,         
    	   breaklines=true,
           keywordstyle=\color{magenta},
        ]
/*R4*/
Select distinct e.EpisodeID, e.SID 
	From Episode e 
    Where Not Exists( 
        Select a.OID 
        From Role a, Personne p 
        	Where p.Numero = a.Numero and a.Nom = p.Nom and a.Prenom = p.Prenom and p.Genre = 'm' and a.OID = e.EpisodeID);

/*R4 bis*/
SELECT distinct EpisodeID
FROM Episode e
WHERE NOT EXISTS(
	SELECT Genre, EpisodeID
	FROM Role INNER Join Personne on  Personne.Nom = Role.Nom AND Personne.Prenom = Role.Prenom AND Personne.Numero  = Role.Numero
	WHERE genre = 'm' AND OID = e.EpisodeID
    );
    
\end{lstlisting}
\subsubsection{Algèbre relationnel}
\subsubsection{Calcul relationnel tuple}

\section{Conclusion}
\end{document}